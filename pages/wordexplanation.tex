% vim:sw=4:ts=4:autoindent:tw=72

% använd detta kommando för att lägga till ett ord
\newcommand{\wordexpl}[2]{%
	\textbf{#1}\hspace*{.2cm}#2\\
}

\chapter*{Word explanation}
\addcontentsline{toc}{chapter}{Word explanation}

% asterisk gör att vi fyller den vänstra kolumnen först
\setlength\columnsep{25pt}
\begin{multicols}{2}
\noindent
Explanation of central concepts in this report. The words might have a different meaning in other contexts.
\vspace*{.1cm}

    \noindent
    \wordexpl{Adjacency matrix}{A matrix defining edges between vertices
    in a graph.}
    \wordexpl{Annotation}{Metadata about a text fragment, picture or other data. In this specific project, annotated nodes refer to nodes where the correct answer is already known.}
    \wordexpl{Attacker}{A mentioned entity that has imposed or imposes a cyber threat towards a target. }
    \wordexpl{Attack vector}{The path or means by which an attack is executed. }
    \wordexpl{Bipartite graph}{A graph with vertices from two adjacent
    sets and edges with one end in each set}
    \wordexpl{Cyber attack}{An attack on infrastructure or information
    systems using methods from computer science.}
    \wordexpl{Dark web}{Crypto-network providing the users with anonymous communication.}
    \wordexpl{Deep web}{websites on the open portion of the internet which are not indexed by search engines.}
    \wordexpl{Directed graph}{A graph where the edges have a distinguished direction leading from one node to another but not in the opposite direction.}
    \wordexpl{Edge}{A link between two nodes in a graph.}
    \wordexpl{Entity}{A person, threat actor, IP address, company or similar.}
    \wordexpl{Exploit}{The method by which a software, data or sequence of commands takes advantage of a vulnerability to cause damage or other unwanted behaviour in a computer software or hardware.}
    \wordexpl{Gh0st RAT}{Malware or more specifically a Trojan horse. RAT is an abbreviation for Remote Administration Tool or Remote Access Trojan.}
    \wordexpl{Malware}{Short for malicious software, which is a software to disrupt, access and monitor systems such as a whole network, a personal computer or an application. There are different sorts of malware, including viruses, Trojan horses and rootkits.}
    \wordexpl{NetFlow}{A reporting software that provides the ability to collect IP network traffic as it crosses an interface. NetFlow is developed by Cisco.}
    \wordexpl{Node}{A vertex in a mathematical graph or a point in a network topology from and/or to which an edge is connected.}
    \wordexpl{Path}{A walk following edges from one node to another. A sequence of nodes and edges.}
    \wordexpl{Reference}{An analysed text fragment harvested from the web. The analysis is performed by Recorded Future and saved in its database.}
    \wordexpl{Simple path}{Path where each visited node is unique.}
    \wordexpl{Subgraph}{A subgraph of a graph $G$ is a smaller graph formed by a subset of the nodes and edges in $G$.}
    \wordexpl{Target}{The entity that is about to be or has been exploited by an attacker or threat actor.}
    \wordexpl{Threat actor}{An entity that imposes a threat towards another entity.}
    \wordexpl{Undirected graph}{A graph where the edges are traversable in both directions, from node $A$ to node $B$ and from node $B$ to node $A$.}
    \wordexpl{Unipartite graph}{Graph consisting of only on class of
    vertices}
    \wordexpl{Vertex}{A node in a graph or a point in a network topology from and/or to which edges are connected. Vertices and edges are the two basic classes of units in a graph.}
    \wordexpl{Vulnerability}{A flaw or weakness in computer security which can be exploited by an attacker or threat actor. }
\end{multicols}

\newpage

% vim:sw=4:ts=4:autoindent:tw=72

\noindent
\begin{comment}
\thesistitle\\
\thesissubtitle\\
\whatthisis\\
\\
\large{%
    Henrik Adolfsson\\
	Josephine Cuellar Andersson\\
}\\
\\
\large{%
	\whereandwhen
}
\vfill
\end{comment}

\begin{center}
    \section*{Abstract}
\end{center}
By representing high dimensional data in graphs, new information and hidden relationships can be extracted. In this master's thesis, a discussion about the transformation from a document database to a graph is given. [to be filled in] 

Furthermore, three case studies were performed on graphs in the cyber threat intelligence domain. The first case treats the classification of Ghost RAT controllers in a graph where IP addresses are represented as nodes. Three different types similarity measures were implemented for classification: local indices, a quasi-local index and a global index. We found that the local indices performed the best, with a high accuracy and a robustness, due to the fact that it is not too sensitive towards the choice of threshold. 

The second case involves the classification of malicious IP addresses divided into 5 classes. Using a one-agains-one SVM based on 6 features, the classifications of Recorded Future's rule based classifier was reproduced with an accuracy of 75 \%. This is not enough to function as a standalone classifier however, there is potential in the method to reach higher accuracies by for instance introducing more features. This method could work as a way of getting insights about the current data, understanding what more rules they should implement in their current classifier or it could work as a complement to their current classifier.

Also, link prediction was applied to a graph consisting of attackers and targets. 


\noindent
\textbf{Key words:} graph, relations, graph analytics, link prediction, node similarity.
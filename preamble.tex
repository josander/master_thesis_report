% vim:sw=4:ts=4:autoindent:tw=72
\documentclass[a4paper,11pt]{report}

\usepackage[utf8]{inputenc}
\usepackage[T1]{fontenc}

% Lista med färgnamn: http://en.wikibooks.org/wiki/LaTeX/Colors
\usepackage[usenames, dvipsnames]{color}

\usepackage[english]{babel}
\usepackage{amsmath}
\usepackage{empheq} % rutor runt ekv
%\numberwithin{equation}{chapter} 
\renewcommand{\frac}[2]{\dfrac{#1}{#2}} % stora fraktioner

\usepackage{units}

\usepackage{graphicx}
% med följande kommando behöver vi ej ange t.ex. images/modell.png utan
% det räcker med modell.png
\graphicspath{{images/}}

\usepackage{bbm}
\usepackage{rotating}
\usepackage{icomma}
\usepackage{comment}
\usepackage{booktabs}
\usepackage{dsfont}
%\usepackage{hyperref}
\usepackage{float}
\usepackage{subfigure}
% se skrivanvisningar för marginal, den anger 3cm
\usepackage[a4paper, margin=3cm]{geometry}
\usepackage[font={small,it}]{caption}
\usepackage{t1enc}
\usepackage{amssymb}
\usepackage{url}
\usepackage{multicol}
\usepackage{wrapfig}

% används i titelsidan för att få samma höjd på miniboxarna
\usepackage{adjustbox}

% används på framsidan för bakgrundsbilden
\usepackage{wallpaper}

% använder copyright-symbolen
\usepackage{textcomp}

% biblatex för referering
\usepackage[sorting=none,natbib=true,url=true]{biblatex}
\bibliography{references}
% alla vetenskapliga artiklar, i princip?, har referens som superscript,
% vi kör också på det:

% använd "tabu" istället för "tabular"
\usepackage{tabu}
\usepackage{longtable}
% radavstånd
\tabulinesep = ^2mm_1mm

% dummy-text
\usepackage{blindtext}

% Typografiska rådet
% 2) 1.5 radavstånd (även om det står 1.3)
\linespread{1.3}
% 8) kompakta fotnoter
\usepackage[para,bottom]{footmisc}

% behövs för \nameref i t.ex. \chapref
\usepackage{hyperref}

\usepackage{appendix}

\usepackage{gensymb} 


% för sidhuvud
\usepackage{fancyhdr}
\fancypagestyle{plain}{
    \fancyhf{} % empty header and footer
    \renewcommand{\headrulewidth}{0pt} % ho header line
    \renewcommand{\footrulewidth}{0pt}% not footer line
    \fancyfoot[C]{\thepage}% like fancy style
}

%%% TODO: byt till "final" vid inlämning
%\usepackage[final]{showlabels}

% kod i kap 5
\usepackage{minted}
\newcommand{\code}[1]{\texttt{#1}}
% http://tex.stackexchange.com/questions/107516/minted-customise-listing-name
\renewcommand\listingscaption{Lista}


%\usepackage{subfig}
%\usepackage{subcaption}


%
% Egna kommandon
%

% mattekommandon
\newcommand{\sgn}{\operatorname{sgn}}

% kommandon för att referera, använd dessa istället för \ref
\newcommand{\Tblref}[1]{Table~\ref{#1}}
\newcommand{\Figref}[1]{Figure~\ref{#1}}
\newcommand{\Secref}[1]{Section \ref{#1}~\emph{\nameref{#1}}}
\newcommand{\Chapref}[1]{Chapter \ref{#1}~\emph{\nameref{#1}}}
\newcommand{\Listref}[1]{List~\ref{#1}}
\newcommand{\tblref}[1]{table~\ref{#1}}
\newcommand{\figref}[1]{figure~\ref{#1}}
\newcommand{\secref}[1]{section \ref{#1}~\emph{\nameref{#1}}}
\newcommand{\chapref}[1]{chapter \ref{#1}~\emph{\nameref{#1}}}
\newcommand{\eqnref}[1]{(\ref{#1})}
\newcommand{\listref}[1]{list~\ref{#1}}

\newcommand*{\figuretitle}[1]{%
    {\centering
    #1
    \par\medskip}
}

% T0DO-kommando:
\newcommand{\todo}[1]{%
	\textbf{{\color{red} TODO: #1}}
}

% titeln på arbetet
\newcommand{\thesistitle}{Revealing relations in highly dimensional temporal data}
\newcommand{\thesissubtitle}{}


% var och när detta arbete gjordes
\newcommand{\whereandwhen}{%
	Department of Physics\\
	\textsc{Chalmers University of Technology}\\
	Göteborg, Sweden 2017\\
}

% vad är det för arbete?
\newcommand{\whatthisis}{%
	Planning Report
}

\usepackage{pgfgantt} % För att göra Gantt-tabell
\usepackage{pdfpages} % för att inkludera pdf-fil
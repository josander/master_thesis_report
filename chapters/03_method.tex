\chapter{Method}
Given the aim of this project and the theory described in the previous chapter, the methodology is now presented. We begin to describe the methodology of this research, where hidden relations have been extracted from Recorded Future's big database to the graph database Neo4j. To validate the method, and understand to what extent a graph representation can be useful, three case studies were selected. The cases are described in detail in \secref{cases}.

\section{Research Methodology}
The aim of this project was to develop a method of analysis that identifies hidden relations in a database. To do this, an inductive research design was used. We started off by studying a few specific cases where the expectation to find patterns leading to a broader generalization. A tentative hypothesis could be formulated which could be tested and later conveyed into a final conclusion.

\section{Research Design}
% To identify relevant cases, interviews will be held
Taking the inductive approach, the first thing we did was to identify important and representative cases. Thus, the research started with interviewing analysts at Recorded Future with the result being a list of questions related to their data that are hard to answer based on the existing database structure and software. The conclusion was that prediction of future events and classification of entities would be beneficial to them. It also helped in understanding of what kind of relations that was interesting to look at from a security analyst perspective.

\section{Validation} 
The final step of the study concerned validation. The developed methods had to be validated in order to be able to evaluate their performances. The validation was performed on a sample of chosen cases, for which the answer was known.

\section{Introduction to the Data Set \label{dataset}}
The data of Recorded Future is comprised of entities, e.g. a country, a person or a company. They are in turn often a component in an ontology, such as Gothenburg being part of Sweden. Another central concept is references, connecting two or more entities. References are a report or text fragment related to a specific event. Furthermore, there is meta data that can be different sorts of data related to entities and events, such as a time interval or type of entity or event.

The data is fetched by queries using Recorded Future's API. The output is a file in JSON format. As previously mentioned, the database contains a lot of information. Thus, it is essential that only a subset of the data is fetched. This leads to the issue of querying the right information and only the right information. To be able to answer a question, we want to have all the necessary information at hand without dealing with too much information. This is based on two reasons; querying information takes time and the more data the more complexity arises.

One difficulty with the data is that some parts are implicit. Since Recorded Future are dealing with natural language processing there are cases when all the information about an event is not available. For instance, if someone on the internet is writing about a cyber attack they might mention an attacker and malware without specifying the target, hence creating implicit data.

Another important aspect is that the data does not reflect the real world but what users on internet find interesting to report. Hence, on one hand there might be some information missing, while there on the other hand may be much data on one single event due to various reasons. An example of the latter case is if Donald Trump's personal computer would be hacked by a hacker representing Anonymous. Due to the popularity of both Trump and Anonymous, it is likely that many people writes about this all over the internet resulting in many references about this specific happening. Thus, the question arises whether there have been multiple attacks or only one. Studying the time of the reports may reveal a lot of information enabling to answer the previous question however there might be cases where there is periodic interest to report about a certain happening. The latter is far more ambiguous.

\section{Graph Representation}
Data modeling with graphs are highly domain dependent and highly dependent on what type of question one is trying to answer. Thus, according to \citet{robinson2013}, the following approach should be taken:
\begin{itemize}
    \item Describe the goals that motivate the model.
    \item Rewrite the goals as questions to ask the domain.
    \item Identify the entities and the relationships that appear in these questions. 
    \item Translate the entities and relationships into Cypher path expressions.
    \item Express the questions we want to ask of our domain as graph patterns using path expressions similar to the ones we used to model the domain. 
\end{itemize}

\section{Extracting Relations From a Database}
% Motivate the need to use a graph database
In a relational database, the extraction of relations are very joint-intensive and hence the performance deteriorates fast when the data set gets larger\cite{robinson2013}. In order to perform analysis using the structure of a graph we needed to find a solution where the limits of querying deep relations in a database would not render the analysis through graphs unusable due to an excessive running time.


\section{Revealing Hidden Relations}

% Motivate the choice of algorithms
The usefulness of the algorithms and measures highly depends on the task \cite{fouss2016algorithms}. It is therefore necessary to perform empirical tests and validate the relevance. 

Algorithms for analysis has been chosen to be as simple as possible yet give a satisfactory result. 


\section{Introduction to the cases \label{cases}}
Below, the three cases which have been studied are presented. They were chosen to give a representative view of the questions one might want to address using a graph representation. Also, the cases have been chosen to show the breadth of use cases that a graph representation could cover.

\subsection{Detection of Gh0st RAT Controllers \label{methodGhost}}
Gh0st RAT (Remote Administration Tool) is a Trojan horse used to hack and take control of computers in real-time. The malware has been used to hack some of the most sensitive computers in the world \citep{cyberspies}.

Our data consists of NetFlow\footnote{NetFlow is a reporting software that provides the ability to collect IP network traffic as it crosses an interface. NetFlow data therefore consist of an undirectional sequence of packets that all share values such as Source IP address, Target IP address and IP protocol. \citep{netflow}} data related to a number of IP addresses that have been identified as Gh0st RAT controllers. The data is annotated such that known RAT controllers have been annotated with the label \textit{Gh0st}. There is also information of the IP traffic such that it is known which IP address has communicated with whom. Thus, in this case our graph model consists of IP addresses as nodes with the relations ``COMMUNICATED'' between them if they have communicated, see \figref{gh0stGraph}. The graph is modelled as an undirected graph, since the RAT controller-slave communication could have been initiated in any direction.

\begin{figure}[h!]
    \centering
    \includegraphics[width=0.75\textwidth]{gh0stRATgraph.png}
    \caption{Graph of IP addresses, visualized in Neo4j.}
    \label{gh0stGraph}
\end{figure}

There are two similar data sets. The first set is based on 21,831 NetFlow records collected during the first months of 2017. In total, the data consist of 3,566 IP addresses where 204 of them have been identified as RAT controllers using packet inspection of responses from these controllers. A topographic view of the network can be seen in figure \ref{ip1}.

\begin{figure}[h!]
    \centering
    \includegraphics[width=0.5\textwidth]{GhostRATs.png}
    \caption{Graph of IP addresses, consisting of 3,566 nodes with 204 known RAT controllers. The graph is visualized using Gephi.}
    \label{ip1}
\end{figure}

The second data set consists of 103,202 NetFlow records collected during December 2016 to Mars 2017. This data set includes 10,091 IP addresses but is far more sparse with \textit{Gh0st}s, containing only 32 identified RAT controllers. 

In this case, the aim was to classify IP addresses, given the knowledge of only a few RAT controllers. As a consequence, we also aimed at identifying IP addresses that are incorrectly annotated as \textit{Non-gh0st}s. Moreover, this study involves a comparison between the seven different measures of structural equivalence. 

The classification has been performed by applying different similarity measures to the network. The applied similarity measures are the Dice coefficient, the Jaccard index, the cosine coefficient, the hub-promoting index and the hub-depressing index. These are all measures of structural equivalence and reflect the similarity based on the local structure, i.e. only the overlap of nearest neighbors.

Furthermore, the Local Path index and the Katz index were studied. Instead of simply studying the information of the nearest neighbors, they also account for more information about the topology. However, the two involve the parameter $\alpha$ that should be properly chosen. To chose it, a fraction of the dataset (20\%) was used to choose the best value of the parameter. 

The similarity was calculated for all nodes, given a small portion (1-5) of known Gh0st RAT controllers, henceforth referred to as reference nodes. Thus, the similarities for one node, in relation to the others, were simply taken as the sum. To classify the node, a threshold was then applied. If the similarity value exceeded the threshold, the node was classified as a RAT controller, i.e. a \textit{Gh0st}.

Once the classification was performed, the classifier using different similarity measures was evaluated. The evaluation was based on the AUC of the precision recall curve and the F$_1$ score. AUC can be interpreted as a value of robustness of the similarity measure while F$_1$ represents the accuracy.

\subsection{Classification of malicious IP addresses \label{methodSVM}}
In this case, we have 102,540 IP addresses from the Recorded Future database. In the graph, the IP addresses are represented as nodes with relations referred to as ``RELATED TO''. The reason the relations have that specific label is because of the nature in which we do not know any more specific details about their co-occurrence. Furthermore, in the graph there are nodes related to malware, attack vectors\footnote{The path or means by which an attack is executed.} and cyber vulnerabilities. These have served as additional information which has been used as information about the neighboring IP addresses.

Recorded Future classifies the IP addresses by giving them a risk score, based on roughly 40 rules. There are 5 different risk classes
\begin{enumerate}
    \item None
    \item Unusual
    \item Suspicious
    \item Malicious
    \item Very Malicious
\end{enumerate}

The histogram of the classifications of the dataset is shown in \figref{hist}. We find the data to have an exponential distribution function, being extremely skewed to the left. Class 1 includes 60\% of the data, class 2 33\%, class 3 4\%, class 4 2.5\% and class 5 0.5\%. This introduces the problem of unbalanced data, discussed in \secref{unbalanced}. To deal with this issue, class weights were introduced such that an error from data in the minority class 5 was penalized higher than the error from data in the majority class 1. The weights were chosen to be inversely proportional to the number of samples in each class in the training set.

\begin{figure}[h!]
    \centering
    \includegraphics[width=0.6\textwidth]{dataHist.eps}
    \caption{Histogram of the IP classifications.}
    \label{hist}
\end{figure}

This case studies whether a graph approach with far less information might be enough to reconstruct Recorded Future's rule based classification. Removing any of their enriched data, we have simply used data found in the original reference. Thus, as a consequence, the following bold question is asked \textit{Is there a way of reconstructing Recorded Future's rule based classifier using a graph representation?} 
To classify the nodes, we have implemented a feature based node classifier. Using features based on topological information, predictions about their criticality was performed.

% Feature extraction
The term feature extraction refers to methods used for constructing node variables from the structure of the graph. These features were then used a input parameters for the classifier. Thus, the performance is highly dependent on the choice of features. In our case, the following features were used
\begin{itemize}
    \item PageRank based on the topology of IP addresses.
    \item Degree based on an undirected graph including only the IP addresses.
    \item Total number of hits in Recorded Future's data base.
    \item Number of unique malwares related to the nearest neighbors.
    \item Number of unique cyber vulnerabilities related to the nearest neighbors.
    \item Number of unique attack vectors related to the nearest neighbors.
\end{itemize}

The features can be divided into two categories. The first includes the PageRank and degree, only using the topology of the graph including only the IP addresses. The second category includes more information from Recorded Future's database, such as the number of hits, co-occurrences with malwares and so on. To study the impact of the different categories of features, classifications were performed including all of the features, only the topology features and finally the features including information regarding the number of hits, malwares, cyber vulnerabilities and attack vectors.

% Choice of classifier
In this case, a SVM classifier has been studied. Many researchers have used SVM for feature classification and found good results \citep{campbell2011}. For instance, \citet{liu2012} showed promising results predicting links in a social network based on a SVM feature classifier. The fact that it was easily extended to a multi-class classifier was also an advantage. Moreover, it is easy to implement and it happens to deal with noise and outliers is a satisfactory fashion using the slack variable, introduced in \secref{svmTheory}.

After choosing the SVM classifier, a kernel had to be chosen. Due to the intermeshed nature of our data, the linear kernel was ruled out and the Radial Basis Function (RBF) kernel was chosen. According to \citet{Hsu10apractical}, the RBF kernel is a good first choice, since the RBF nonlinearly maps the data into a higher dimensional space. In addition, it has the advantage of including fewer parameters than the polynomial kernel implying a lower complexity.

After choosing the SVM classifier with a RBF kernel, the parameters $C$ and $\gamma$ had to properly be chosen. In order to do so, a grid search was performed. Various pairs of $C$ and $\gamma$ was tried and evaluated based on cross-validational accuracy of 10\% of the data. This was to avoid the pitfall of overfitting to the only dataset available. Once again in accordance with the recommendation of \citet{Hsu10apractical}, an exponentially growing sequence of parameters were studied where $C\in\{10^{-2},5\cdot10^{-2},10^{-1},5\cdot10^{-1},10^{0},...,10^{3}\}$ and $\gamma\in\{10^{-3},5\cdot10^{-3},10^{-2},5\cdot10^{-2},...,10^{2}\}$. The evaluation of the classifier was based on the accuracy, referring the the fraction of correct predictions. The set of parameters were chosen by maximizing the accuracy while making sure the predictions included all five classes. Since three different sub-cases were studied, each including a different set of features, the grid search was performed for each individual case. 

The SVM takes a matrix of features as input. As mentioned in \secref{scale}, it is preferable to rescale the feature vectors to avoid that some features are dominated by others. The rescaling was performed for each feature, in the range $[0,1]$, keeping the interrelation between elements in the feature vector. This is in accordance with the recommendation of \citet{Hsu10apractical}.

% Cross-validation including split into training and test set
After implementing the classifier, it needed to be validated. A 10-fold cross-validation was applied with 50 repetitions, in order to get a good statistical foundation.

Because of the skewness of the dataset, a pseudorandom classifier, based on a uniform distribution, was also implemented.  The results from it was then used in order to evaluate the SVM results. 

\subsection{Cyber Attacks}\label{cyberattacks}
Recorded Future's database contain records that represents an event called cyberattack. The cyberattack event entity is based on information found on the open internet as well as the dark/deep web. The event entities may contain information such as the time when the attack is believed to have taken place, the author of the source of information, possible attackers and target as well as vulnerabilities and methods used.

The whole set of cyberattacks is quite large. To predict future cyberattacks by the method of link prediction in bipartite graphs (described in \secref{sec:plp}), we used only a subset of the cyberattack event entities containing only the entities with a mention of an attacker and a target. The size of the subset containing all such references was 1,435,073. The number of cyberattacks that was indexed as taken place in the last 30 days when the report was written, was 36,297 which indicates how fast the subset we are using for analysis is growing.

The cyberattacks in the data set were represented in a bipartite graph with the two adjacent sets being attackers and targets. It is worth mentioning that there are several cases where an entity appears as an attacker in one cyberattack and later as a target in another cyberattack. This goes against the definition of the bipartite graph. The solution was to create two different vertices representing in one case an attacker and in the other a target for such entities. The number of references for each attacker-target pair was used as the weight of the edge between the attacker and the target of the attack in the graph.

The running time of the PLP-algorithm by \citet{plp}, described in \secref{sec:plp} is $O(m)$ with m being the size of the smallest of the two adjacent sets of nodes in the bipartite graph. The subset we used contained less attackers then targets, hence the run time was $O(a)$ where $a$ is the number of attackers.

Even though PLP is quite efficient, the run time could be reduced even further if link prediction could be made with good accuracy using only the most recent cyberattacks. We therefore investigated how the prediction rate and the precision was affected by different length of the time periods used for prediction. We were also interested in finding out what could be said about when a predicted attack will take place and therefore we also varied the length of the time period used for testing the predictions.

To get good statistical foundation we repeated each experiment 10 times by randomly picking a start date for the prediction data set. We then computed the average values together with the standard deviations.

To evaluate how well PLP was able to predict future cyberattacks we primarily used two values, AUC and prediction rate.

\paragraph{AUC}\label{plp:auc}
We wanted to find out how often the predictive index was higher for attacks that existed in the test set than for attacks that were not in the test set. We therefore measured the AUC as
$$
  AUC = \frac{A+0.5C}{n},
$$
where A is the number of times where the existing attacks had a higher index, C is the number of times the index was the same and $n$ is the number of comparisons. We see that if the existing attacks has a higher index in all comparisons the precision is 1, if all compared index is equal the precision is 0.5, and if the index was higher for all non-existing attacks the precision is 0.

\paragraph{Prediction Rate}\label{plp:predict_rate}
Since PLP only gives a predictive index for the candidate node pairs it is reasonable to evaluate how many of the attacks in the test set that were given prediction index. We computed the prediction rate simply as
$$
  \frac{p}{m}
$$
where $p$ is the number of attacks in the test set that was given an index by PLP and $m$ is the number of attacks with unique pairs of attackers and targets.

The algorithm is naturally limited by that it is not able to predict attacks where the attack-target pair exist in an attack already in the prediction set. The algorithm is also not able to predict attacks involving attackers or targets not in existence in the prediction set. Based on this it is interesting to see how many of the attacks in the test set are attacks that consist of an attacker-target pair not already in the prediction set but with each of the attacker and target individually in the prediction set.



\newpage 

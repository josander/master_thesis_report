\chapter{Results}
In this chapter our results are presented. Starting off with a description f the graph database, we then move on to the three different cases. 

\section{Graph Database}
% Power of graph databases
We are dealing with connected data, where choosing a graph database have two advantages according to \citet{robinson2013}: performance and flexibility. In a relational database, the join-intensive query performance deteriorates as the dataset gets larger while, with a graph database the performance remains relatively constant. This is due to the fact that a query is only localized to a fraction of the graph making the run time proportional to the size of the sub-graph related to the query rather than the overall graph containing all available data.

Furthermore, there is the aspect of flexibility. Graphs are naturally additive \cite{robinson2013} such that new sub-graphs can be added containing more nodes and edges and new relations can be introduced as the understanding of the dataset grows. This have a positive implication for the analysts that does not have to model the domain in exhaustive detail from the start but can add information as time progress.

Due to the above reasoning we opted to use Neo4j, which is a graph database supporting some basic analysis through their query language Cypher.

\section{Gh0st RAT Controllers}
In this section, we present the AUC values and the maximal measured F$_1$ score for the different indices and different numbers of reference nodes that the similarity is based on. We first present the results for the first dataset and then move on the the second dataset. Since cross validation is performed, 20 trials has been carried out.

\textbf{The first dataset.} The result for the first dataset in the RAT controller case is given in table \ref{aucIndex}. We find small variance in the results for the local indices. The quasi-local Local Path index and the global Katz index perform wore than the local ones. 

\begin{table}[h!]
    \centering
    \caption{AUC and F$_1$ score for different number of reference nodes. The AUC is calculated from the precision-recall curve. The F$_1$ score represents the average accuracy between the 20 runs.}
    \begin{tabular}{|c||c|c||c|c||c|c||c|c||c|c||}
      \hline
      \multirow{2}{*}{~} 
            & \multicolumn{2}{c||}{1}
            & \multicolumn{2}{c||}{2}
            & \multicolumn{2}{c||}{3}
            & \multicolumn{2}{c||}{4}
            & \multicolumn{2}{|c|}{5} \\             \cline{2-11}
      ~     &AUC&F$_1$&AUC&F$_1$&AUC&F$_1$&AUC&F$_1$&AUC&F$_1$ \\ \hline
    Dice    &0.8995 & 0.89 & 0.9363 &0.94 &0.9554&0.97 & 0.9627 &0.98&0.9658 & 0.98 \\
    Jaccard &0.8978 & 0.89 & 0.9367 &0.94 &0.9552&0.97 & 0.9616 &0.98&0.9654 & 0.98 \\
    Cosine  &0.8943 & 0.89 & 0.9337 &0.94 &0.9518&0.97 & 0.9591 &0.98&0.9621 & 0.98 \\
    ProHub  &0.8850 & 0.89 & 0.9244 &0.94 &0.9435&0.97 & 0.9478 &0.98&0.9434 & 0.98 \\
    DeHub   &0.8987 & 0.89 & 0.9378 &0.94 &0.9563&0.97 & 0.9631 &0.98&0.9664 & 0.98 \\
    LP      & 0.7930 & 0.78 & 0.9283 & 0.96 & 0.9279 & 0.96 & 0.9426 & 0.98 & 0.9206 & 0.98 \\ 
    Katz    & 0.1888 & 0.29 & 0.1960 & 0.28 & 0.1949 & 0.26 & 0.1759 & 0.28 & 0.1850 & 0.27 \\ \hline
    \end{tabular}
    \label{aucIndex}
\end{table}


With all the described indices and the mentioned parameters, all RAT controllers were identified properly, with a precision of 100 \% and a recall of 100 \%.

Futhermore, there were three IP addresses that were constantly classified as RAT controllers, while they in fact were annotated as \textit{Non-ghost}s. In XX \% of the times, 

% In how many of the cases did we classify the three IP addresses as RAT controllers?
    
\textbf{The second dataset.} The second data set was much more sparse of RAT controllers. Of 10091 nodes, only 32 were identified as RAT controllers. Performing the exact same analysis on this data set as the previous one gave us the result shown in table \ref{aucIndex2}.

\begin{table}[h!]
    \centering
    \caption{AUC and F$_1$ score for different number of reference nodes. The AUC is calculated from the precision-recall curve. The F$_1$ score represents the average accuracy between the 20 runs.}
    \begin{tabular}{|c||c|c||c|c||c|c||c|c||c|c||}
      \hline
      \multirow{2}{*}{~} 
            & \multicolumn{2}{c||}{1}
            & \multicolumn{2}{c||}{2}
            & \multicolumn{2}{c||}{3}
            & \multicolumn{2}{c||}{4}
            & \multicolumn{2}{|c|}{5} \\             \cline{2-11}
      ~     &AUC&F$_1$&AUC&F$_1$&AUC&F$_1$&AUC&F$_1$&AUC&F$_1$ \\ \hline
    Dice    & 0.9696 & 0.98 & 0.9680 & 1.00 & 0.9531 & 1.00 & 0.9375 & 1.00 & 0.9219 & 1.00 \\
    Jaccard & 0.9696 & 0.98 & 0.9680 & 1.00 & 0.9531 & 1.00 & 0.9375 & 1.00 & 0.9219 & 1.00 \\
    Cosine  & 0.9696 & 0.98 & 0.9680 & 1.00 & 0.9531 & 1.00 & 0.9375 & 1.00 & 0.9219 & 1.00 \\
    ProHub  & 0.9696 & 0.98 & 0.9680 & 1.00 & 0.9531 & 1.00 & 0.9375 & 1.00 & 0.9297 & 1.00 \\
    DeHub   & 0.9696 & 0.98 & 0.9680 & 1.00 & 0.9531 & 1.00 & 0.9375 & 1.00 & 0.9219 & 1.00 \\ 
    LP      & 0.7930 & 0.78 & 0.9283 & 0.96 & 0.9279 & 0.96 & 0.9426 & 0.98 & 0.9206 & 0.98 \\ 
    Katz    & 0.1888 & 0.29 & 0.1960 & 0.28 & 0.1949 & 0.26 & 0.1759 & 0.28 & 0.1850 & 0.27 \\ \hline
    \end{tabular}
    \label{aucIndex2}
\end{table}



\FloatBarrier
\section{Classification of malicious IP addresses}

When including all the six features, the best accuracy was found for the parameters $C=100$ and $\gamma=1$. The results for the SVM, using six features, are presented in table \ref{IpRes}. The table presents the average over 100 trials and the standard deviation with a confidential level of 95 \%. The SVM classifier, based on all 6 input features, performed prediction with an accuracy of 50 \%. Futhermore, ... of the predicted classes were higher than Recorded Future's while ... \% were lower. 

\begin{table}[h!]
    \centering
    \caption{Results from the SVM classifier using six features. The table shows the fraction of correct predictions (accuracy), the fraction of higher predictions and the fraction of lower predictions in relation to Recorded Future's actual classification.}
    \begin{tabular}{|c|c|}
    \hline
        \multicolumn{2}{|c|}{SVM with 6 features}\\ \hline
        Correct predictions & 0.200227 $\pm$0.002237 \\
        Higher predictions  & 0.699423 $\pm$0.002294\\
        Lower predictions   & 0.100350 $\pm$0.001660\\ \hline
    \end{tabular}
    \label{IpRes}
\end{table}

The results from including only the two topological features are presented in table \ref{IpRes2Feat}. The features were PageRank and degree of the node. The table presents the average over 100 trials and the standard deviation with a confidential level of 95 \%. The SVM classifier performed prediction with an accuracy of 50 \%. Futhermore, ... of the predicted classes were higher than Recorded Future's while ... \% were lower. 

\begin{table}[h!]
    \centering
    \caption{Results from the SVM classifier using two features. The table shows the fraction of correct predictions (accuracy), the fraction of higher predictions and the fraction of lower predictions in relation to Recorded Future's actual classification.}
    \begin{tabular}{|c|c|}
    \hline
        \multicolumn{2}{|c|}{SVM with 2 features}\\ \hline
        Correct predictions & 0.200227 $\pm$0.002237 \\
        Higher predictions  & 0.699423 $\pm$0.002294\\
        Lower predictions   & 0.100350 $\pm$0.001660\\ \hline
    \end{tabular}
    \label{IpRes2Feat}
\end{table}

The results from including only four features are presented in table \ref{IpRes4Feat}. The features included information about the number of hits and the neighbours co-occurrence with malwares, cyber vulnerabilities and attack vectors. The table presents the average over 100 trials and the standard deviation with a confidential level of 95 \%. The SVM classifier performed prediction with an accuracy of 50 \%. Futhermore, ... of the predicted classes were higher than Recorded Future's while ... \% were lower. 

\begin{table}[h!]
    \centering
    \caption{Results from the SVM classifier using four features. The table shows the fraction of correct predictions (accuracy), the fraction of higher predictions and the fraction of lower predictions in relation to Recorded Future's actual classification.}
    \begin{tabular}{|c|c|}
    \hline
        \multicolumn{2}{|c|}{SVM with 4 features}\\ \hline
        Correct predictions & 0.200227 $\pm$0.002237 \\
        Higher predictions  & 0.699423 $\pm$0.002294\\
        Lower predictions   & 0.100350 $\pm$0.001660\\ \hline
    \end{tabular}
    \label{IpRes4Feat}
\end{table}

For a pseudorandom classifier, the results can be found in table \ref{randClass}. We have found an accuracy of 20 \%, which was to be expected. However, 70 \% of the predictions gave a higher risk classification and 10 \% gave a lower risk classification.

\begin{table}[h!]
    \centering
    \caption{Results from the a random classifier, based on a uniform distribution. The table shows the fraction of correct predictions (accuracy), the fraction of higher predictions and the fraction of lower predictions in relation to Recorded Future's actual classification.}
    \begin{tabular}{|c|c|}
    \hline
        \multicolumn{2}{|c|}{Random classifier}\\ \hline
        Correct predictions & 0.200227 $\pm$0.002237 \\
        Higher predictions  & 0.699423 $\pm$0.002294\\
        Lower predictions   & 0.100350 $\pm$0.001660\\ \hline
    \end{tabular}
    \label{randClass}
\end{table}

\FloatBarrier
\section{Prediction of Future Cyber Attacks}
The results of predicting future cyber attacks using PLP on the data set described in \secref{cyberattacks} is shown in \figref{fig:plp_results} for different length of the period used for prediction and for different lengths of the period used for testing.
\begin{table}[!ht]
    \centering
    \begin{tabular}{|c|c|c|c|}
    \hline
        Prediction Months & Test Months & AUC            & Prediction Rate  \\ \hline
        24                & 24          & 0.941(0.0038)  & 0.251(0.0085)    \\
        24                & 18          & 0.943(0.0039)  & 0.252(0.0073)    \\
        24                & 12          & 0.948(0.0039)  & 0.255(0.013)     \\
        24                & 6           & 0.954(0.0073)  & 0.243(0.024)     \\
        24                & 3           & 0.951(0.0060)  & 0.233(0.016)     \\
        24                & 2           & 0.953(0.015)   & 0.220(0.011)     \\
        24                & 1           & 0.953(0.026)   & 0.213(0.027)     \\ \hline
    \end{tabular}
    \caption{\label{fig:plp_results} Results for the PLP algorithm used on the cyber attack bipartite graph. AUC and prediction rate for different lengths (in months) of the time period used for prediction as well as the length used for the time period for testing. AUC was calculated according to the description in section \ref{plp:auc} and prediction rate according to \ref{plp:predict_rate}.}
\end{table}

We find that the highest prediction rate is given when the prediction time period and test period is the longest but is almost as good for prediction lengths of 6 months and test periods of 1 months. The AUC is consistent but deviates more when the length of the test period is reduced.

The maximum possible prediction rate varied between 60\%-70\%.


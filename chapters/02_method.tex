\chapter{Method}
Given the aim of this project, we now present the methodology. This leads to an introduction to Recorded Future's data and expected challenges related to the data.

\section{Research Methodology}
The aim of this project is to develop a method of analysis that identifies hidden relations in a document database. To perform this, an inductive research design will be taken. Starting off by studying a few specific cases, the expectation is to find some pattern leading to a broader generalization. From there, tentative hypothesis will be formulated which later will be tested and hopefully conveyed into a final conclusion.

\section{Research Design}
% To identify relevant cases, interviews will be held
Taking the inductive approach, the first thing to do is to identify important and representative cases. Thus, the research starts with interviewing analysts at Recorded Future with the result being a list of questions related to their data that are hard to answer based on the existing database and software. This will give us information about what kind of relations that are interesting to look at from a security analyst perspective.

% Suitable choices of analysis 
In parallel, a literature study on the field of network science will be performed. This will lay the foundation of network analysis know how and will help us to answer the questions given by the analysts. 

\section{Validation} 
The final step of the study concerns validation. The developed method must be validated to tell whether it is performing well or not. The validation will be performed on a sample of chosen cases, for which the answer is known. The answer could be figured out by for instance perform extensive analysis in multiple steps with the existing database and software.

\section{Introduction to the Data Set}
The data of Recorded Future is comprised of entities, e.g. a country, a person or a company. They are in turn often a component in an ontology, such as Stockholm being part of Sweden. Another central concept is references, often connecting two entities. References are a report or text fragment related to a specific event. Furthermore, there is metadata that can be different sorts of data related to entities and events, such as a time interval or type of entity or event. %Thus, the data is multidimensional, however it can by reduction be represented by a two dimensional network, if for instance only a certain time interval is chosen.

The data is fetched by queries using Recorded Future's API. The output is a file in JSON format. As previously mentioned, the database contains a lot of information. Thus, it is essential that only a subset of the data is fetched. This leads to the issue of querying the right information and only the right information. To be able to answer a question, we want to have all the necessary information at hand without dealing with too much information. This is based on two reasons; querying information takes time and the more data the more complexity arises.

% Often implicit sets of data, with information about an attack and attacker but not target or similar
One difficulty with the data is that some parts are implicit. Since Recorded Future are dealing with natural language processing there are cases when all the information about an event is not available. For instance, if someone on the Internet is writing about a cyber attack they might mention an attacker and malware without specifying the target, hence creating implicit data. 

Another important aspect is that the data does not reflect the real world but what users on Internet find interesting to report. Hence, on one hand there might be some information missing, while there on the other hand may be much data on one single event due to various reasons. An example of the latter case is if Donald Trump's personal computer would be hacked by a hacker representing Anonymous. Due to the popularity of both Trump and Anonymous, it is likely that many people writes about this all over the Internet resulting in many references about this specific happening. Thus, the question arises whether there have been multiple attacks or only one. Studying the time of the reports may reveal a lot of information enabling to answer the previous question however there might be cases where there is periodic interest to report about a certain happening. The latter is far more ambiguous.

\newpage 

\begin{comment}
Perform a literature study on the field of networks analysis with the purpose of building a foundation of network analysis know how.
Research what algorithms are available for extracting hidden or chained relations in a database with low running time complexity.
Gather information about what kind of relations are interesting to look at from an security analyst perspective.
Perform security analysis of some special representative collections of threat intelligence questions.
\end{comment}
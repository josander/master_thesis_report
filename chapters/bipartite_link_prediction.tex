The recent years growth of large social networks such as those from large consumer sites as Amazon and social sites as Facebook and LinkedIn have yielded an increasing interest in network analysis. A network graph might only disclose a part of a large picture or might by changing with time, therefore it is useful to develop methods for link prediction.

The aim of link prediction analysis is to predict future edges in a temporal graph or discover hidden links in a graph by considering the topological structure of the graph. Many recent approaches that aims at predicting edges in social networks are similarity based where a vertex pair index is used to indicate the similarity of the two nodes. The index value is used to make predictions about the likelihood of future links. There are many different attributes that can be used to indicate similarity. The similarity index can either by global, local or quasi-local. The methods that are local only consider the information about the closest neighborhood. Example of methods that are local are Common Neighbors, Jaccard index, Hub Promoted and Resource Allocation\cite{linkpredict}. The global methods use topological information about the whole graph and include Katz, and Matrix Forest Index\cite{linkpredict}. The quasi-local methods does not require as information about the whole graph but requires more than the local ones. Another type of similarity method is random walk methods including SimRank, Cos+ and Average Commute Time \cite{linkpredict}. There has also been a lot of work where Machine Learning strategies have been used for link prediction methods \cite{mlpredict1,mlpredict2,mlpredict3,mlpredict4,mlpredict5,mlpredict6,mlpredict7}.
